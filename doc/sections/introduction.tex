\subsection{Context of the problem}
\indent
One of the major environmental concerns of our time is the increasing consumption of Earth's resources to sustain our way of life. As more and more nations make the climb up from agricultural to industrial nations, their standard of life will improve, which will mean that more and more people will be competing for the same resources. While NASA spinoffs and other inventions can allow us to be more thrifty with Earth's resources, we nevertheless must come to grips with the problem that humanity is currently limited to one planet. In addition, the renowned astrophysicist, Stephin Hawking, said he fears mankind is at its greater risk ever, facing war, resource depletion and overpopulation and its future ``must be in space'' if it is to survive \cite{telegraph}. As a whole, Space colonies could be the answer to these problems. While some studies are searching for the suitable planets as migration destinations, comparing the planetary habitabilities\cite{Grenfell:2009imba}, some are investigating suitable ways to reach those destinations, such as interstellar travels\cite{Crawford:2011ggba}. If we can solve the medical problems posed by microgravity (also called weightlessness) and the high levels of radiation to which the astronauts would be exposed after leaving the protection of the Earth's atmosphere, the colonists would mine the Mars or the minor planets and build beamed power satellites like the power plants on the Earth\cite{nasa}. The colonists could also take adavantage of the plentiful raw materials, unlimited solar power, vaccuum, and microgravity in other ways to create products that we cannot while inside the cocoon of Earth's atmosphere and gravity. In addition to potentially replacing our current Earth-polluting industries, these colonies may also help our environment in other ways. Since the colonists would inhabit completely isolated manmade environments, they would refine our knowledge of the Earth's ecology. Together with all these fantastic plans on every aspect, a new plan to organize our new community should also be added to the whole blueprint to make our planetary society a better one.

In fact, this vision, which was purely science fiction for years and years, has caught the imagination of the public for a very long time. Human beings have long been dreaming about settling down on a new planet for a better living, which will provide them with substantial income, adequate education and improved equality. Lots of science fiction novels have described the life on another planet, taking [Red Mars] by Kim Stanley Robinson and [The Martian] by Andy Weir as examples. As we can see easily, most novelists have huge passions in setting Mars as the destination of human migrations, the neighbour planet of Earth potentially with water molecules on it. So let's take the time machine to travel to the year of 2095 and make this dream come true, leading to the establishment of the new organization known today as Laboratory of Interstellar Financial \& Exploration Policy (LIFE). 

LIFE has completed a series of short-term planned living experiments on Mars and has provided new technologies for the first wave of migration, called Population Zero, will include 10,000 people. UTOPIA: 2100, the project launched by the LIFE agency, with the the goal of creating an optimal workforce for the 22nd century to give all people the greatest quality of life with a vision of sustainability for the next 100 years, is exactly the plan to organize our new community into a better one. We, the three-person policy modeling team, are part of the group of advisors and policy makers in the International Coalition on Mars (ICM) government. Our job is to develop a policy model and make recommendations that will create a sustainable life-plan and will make the living experience on Mars in the year 2100 even better than the Earthly one in the current year of 2095.

For Population Zero to achieve optimal conditions in many workforce and social living factors, the goal of our model should be creating a sustainable society with both maximal economic output (GDP) and happiness in the work place for Population Zero citizens. As a result, we will mainly consider three balancing factors, which are:
\begin{description}
\item [Income]: Ensure adequate compensation so that all people can afford fundamental necessities (shelter, food, clothes).
\item [Education]: Provide high quality education that prepares citizens for the needs and challenges of the 22nd Century.
\item [Equality]: Improve the retention of women in the workforce, particularly in fields where they have been underrepresented or discriminated against on Earth.
\end{description}